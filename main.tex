\documentclass[17pt,a4paper]{extreport}

% ===== НАСТРОЙКИ ДЛЯ КИРИЛЛИЦЫ =====
\usepackage[utf8]{inputenc}
\usepackage[T2A]{fontenc}
\usepackage[english, russian]{babel}

% ===== ДОПОЛНИТЕЛЬНЫЕ ПАКЕТЫ =====
\usepackage{cmap}
\usepackage{mathtext}
\usepackage{amsmath,amsfonts,amssymb}
\usepackage{graphicx}
\usepackage{indentfirst}
\usepackage{geometry}
\geometry{left=2cm,right=1.5cm,top=2cm,bottom=2cm}
\usepackage[hidelinks]{hyperref}

% ===== НАСТРОЙКИ ДОКУМЕНТА =====
\frenchspacing
\usepackage{parskip}
\setcounter{secnumdepth}{-1}

\begin{document}
\tableofcontents
\newpage

\section{1. Понятие двойного интеграла}

Будем считать, что все области двумерного пространства являются квадрируемыми (имеющими площадь), а границы областей~--- непрерывными, гладкими, замкнутыми линиями. 

Отрезок, соединяющий любые две точки границы области и принадлежащий этой области, называется \emph{хордой}. Наибольшая из хорд~--- это \emph{диаметр} области.

Пусть $D \subset \mathbb{R}^2$, функция $f(x,y)$ задана на области $D$. Разобьём $D$ на $n$ маленьких частиц $(D_i)$, где $S_{\Delta i} = \Delta \sigma_i$~--- площадь каждой частицы. На каждой площадке выберем произвольным образом точку $M_i (\alpha_i, \beta_i)$.

Составим интегральную сумму:
\[
S_n = \sum_{i=1}^{n} f(\alpha_i, \beta_i) \Delta \sigma_i.
\]
Обозначим через $\lambda = \max \Delta \sigma_i$ максимальный диаметр частичных областей. Если при $\lambda \to 0$ существует предел $\lim_{\lambda \to 0} S_n$, то он называется \emph{двойным интегралом} функции $f(x, y)$ по области $D$ и обозначается:
\[
\iint\limits_D f(x,y)\,dx\,dy
\]

Если этот предел существует и конечен, то функция $f(x,y)$ называется \emph{интегрируемой} в области $D$.

\newpage

\section{2. Задачи, приводящие к понятию двойного интеграла}

\subsection*{1) Механическое приложение}

Если в области $D$ распределено вещество с плотностью $\rho = f(x,y)$, то масса вещества вычисляется по формуле:
\[
m = \iint\limits_D f(x,y)\,dx\,dy
\]

\subsection*{2) Геометрическое приложение}

Пусть в области $D$ задана положительная непрерывная функция $z = f(x,y)$. Тогда объём тела, лежащего в плоскости $Oxy$ и ограниченного сверху поверхностью $z = f(x,y)$, вычисляется по формуле:
\[
V = \iint\limits_D f(x,y)\,dx\,dy
\]

Двойной интеграл выражает объём цилиндрического тела, образующие которого параллельны оси $Oz$ и которое ограничено снизу областью $D$, а сверху~--- поверхностью $z = f(x,y)$.

\end{document}