\documentclass[17pt,a4paper]{extreport}

% ===== НАСТРОЙКИ ДЛЯ КИРИЛЛИЦЫ =====
\usepackage[utf8]{inputenc}
\usepackage[T2A]{fontenc}
\usepackage[english, russian]{babel}

% ===== ДОПОЛНИТЕЛЬНЫЕ ПАКЕТЫ =====
\usepackage{cmap}
\usepackage{mathtext}
\usepackage{amsmath,amsfonts,amssymb}
\usepackage{graphicx}
\usepackage{indentfirst}
\usepackage{geometry}
\geometry{left=1cm,right=1cm,top=2cm,bottom=2cm}
\usepackage[hidelinks]{hyperref}

% ===== НАСТРОЙКИ ДОКУМЕНТА =====
\frenchspacing
\usepackage{parskip}
\setcounter{secnumdepth}{-1}

\begin{document}
\tableofcontents
\newpage

\section{1. Понятие двойного интеграла}

Будем считать, что все области двумерного пространства являются квадрируемыми (имеющими площадь), а границы областей~--- непрерывными, гладкими, замкнутыми линиями. 

Отрезок, соединяющий любые две точки границы области и принадлежащий этой области, называется \emph{хордой}. Наибольшая из хорд~--- это \emph{диаметр} области.

Пусть $D \subset \mathbb{R}^2$, функция $f(x,y)$ задана на области $D$. Разобьём $D$ на $n$ маленьких частиц $(D_i)$, где $S_{\Delta i} = \Delta \sigma_i$~--- площадь каждой частицы. На каждой площадке выберем произвольным образом точку $M_i (\alpha_i, \beta_i)$.

Составим интегральную сумму:
\[
S_n = \sum_{i=1}^{n} f(\alpha_i, \beta_i) \Delta \sigma_i.
\]
Обозначим через $\lambda = \max \Delta \sigma_i$ максимальный диаметр частичных областей. Если при $\lambda \to 0$ существует предел $\lim_{\lambda \to 0} S_n$, то он называется \emph{двойным интегралом} функции $f(x, y)$ по области $D$ и обозначается:
\[
\iint\limits_D f(x,y)\,dx\,dy
\]

Если этот предел существует и конечен, то функция $f(x,y)$ называется \emph{интегрируемой} в области $D$.

\newpage

\section{2. Задачи, приводящие к понятию двойного интеграла}

\subsection{1) Механическое приложение}

Если в области $D$ распределено вещество с плотностью $\rho = f(x,y)$, то масса вещества вычисляется по формуле:
\[
m = \iint\limits_D f(x,y)\,dx\,dy
\]

\subsection{2) Геометрическое приложение}

Пусть в области $D$ задана положительная непрерывная функция $z = f(x,y)$. Тогда объём тела, лежащего в плоскости $Oxy$ и ограниченного сверху поверхностью $z = f(x,y)$, вычисляется по формуле:
\[
V = \iint\limits_D f(x,y)\,dx\,dy
\]

Двойной интеграл выражает объём цилиндрического тела, образующие которого параллельны оси $Oz$ и которое ограничено снизу областью $D$, а сверху~--- поверхностью $z = f(x,y)$.

\newpage

\section{3. Основные свойства двойного интеграла}

\subsection{1) Аддитивность}

Если $f(x,y)$ интегрируема в области $D$, то она интегрируема в любой подобласти $D' \subset D$ ($D$, $D'$~--- квадрируемые области).

Если $f(x,y)$ интегрируема на $D_1$ и $D_2$, то $f(x,y)$ интегрируема на $D = D_1 \cup D_2$.

Если $D = D_1 \cup D_2$, где $D_1$ и $D_2$~--- квадрируемые множества без общих внутренних точек, то:
\[
\iint\limits_D f(x,y)\,dx\,dy = \iint\limits_{D_1} f(x,y)\,dx\,dy + \iint\limits_{D_2} f(x,y)\,dx\,dy
\]

\subsection{2) Линейность}

Пусть $f(x,y)$, $g(x,y)$ интегрируемы на $D$, $\alpha, \beta \in \mathbb{R}$. Тогда $\alpha \cdot f(x,y) + \beta \cdot g(x,y)$ интегрируема на $D$ и:
\[
\iint\limits_D [\alpha \cdot f(x,y) + \beta \cdot g(x,y)]\,dx\,dy = \alpha \iint\limits_D f(x,y)\,dx\,dy + \beta \iint\limits_D g(x,y)\,dx\,dy
\]

\subsection{3) Монотонность}

Если $f(x,y)$, $g(x,y)$ интегрируемы на $D$ и $f(x,y) \leq g(x,y)$ для любого $(x,y) \in D$, то:
\[
\iint\limits_D f(x,y)\,dx\,dy \leq \iint\limits_D g(x,y)\,dx\,dy
\]

\subsection{4) Интегрируемость произведения}

Если $f(x,y)$, $g(x,y)$ интегрируемы на $D$, то $f(x,y) \cdot g(x,y)$ также интегрируема на $D$.

\subsection*{5) Теорема о среднем}

Пусть:
\begin{enumerate}
\item $f(x,y)$, $g(x,y)$ --- интегрируемы на $D$
\item $g(x,y) \geq 0$ на $D$
\item $m = \inf\limits_D f(x,y)$, $M = \sup\limits_D f(x,y)$
\end{enumerate}

Тогда существует $\mu \in [m, M]$ такое, что:
\[
\iint\limits_D f(x,y) g(x,y)\,dx\,dy = \mu \iint\limits_D g(x,y)\,dx\,dy
\]

\newpage

\section{4. Вычисление двойного интеграла в декартовой системе координат}

Пусть $D \subset \mathbb{R}^2$ и задана непрерывная функция $f(x,y)$. Предположим, что $D$ ограничена непрерывной замкнутой кривой $\partial D = l$, которую любая вертикальная или горизонтальная прямая пересекает не более чем в двух точках.

Предположим, что $D$ расположена в прямоугольнике:
\[
\begin{cases}
a \leq x \leq b \\
c \leq y \leq d
\end{cases}
\]

Прямые $x = a$ и $x = b$ точками касания делят границу на две части: $l = l_1 \cup l_2$. Предположим, что:
\[
l_1: y = y_1(x); \quad l_2: y = y_2(x)
\]
тогда двойной интеграл вычисляется по формуле:
\[
\iint\limits_D f(x,y)\,dx\,dy = \int_a^b dx \int_{y_1(x)}^{y_2(x)} f(x,y)\,dy
\]

Второй случай (интегрирование в обратном порядке): пусть прямые $y = c$ и $y = d$ точками касания делят границу $l$ на две части: $l = l_3 \cup l_4$ так, что:
\[
l_3: x = x_1(y); \quad l_4: x = x_2(y)
\]
тогда двойной интеграл вычисляется по формуле:
\[
\iint\limits_D f(x,y)\,dx\,dy = \int_c^d dy \int_{x_1(y)}^{x_2(y)} f(x,y)\,dx
\]

Интегралы в правых частях первого и второго случая называются \emph{повторными}.

\section{5. Криволинейные координаты на плоскости}

Пусть двумерная квадрируемая область $G \subset \mathbb{R}^2$ задана функциями:
\[
x = x(u,v), \quad y = y(u,v), \quad (u,v) \in G
\]
Эта система функций определяет отображение области $G$ в область $D \subset \mathbb{R}^2$.

Предположим, что отображение удовлетворяет следующим условиям:
\begin{enumerate}
\item Оно взаимно однозначно и имеет обратное отображение:
  \[
  u = u(x,y), \quad v = v(x,y), \quad (x,y) \in D
  \]
\item Функции $x = x(u,v)$, $y = y(u,v)$ непрерывно дифференцируемы в области $G$, причём якобиан отображения:
  \[
  J = \frac{\partial(x,y)}{\partial(u,v)} = 
  \begin{vmatrix}
  x'_u & x'_v \\
  y'_u & y'_v
  \end{vmatrix} \neq 0
  \]
\end{enumerate}

В силу непрерывности частных производных функций, якобиан сохраняет свой знак в области $G$. Отображение устанавливает взаимно однозначное соответствие между двумя областями, а также между кусочно-гладкими кривыми в областях $D$ и $G$.

Линия $u = u_0$ в $G$ соответствует гладкой кривой $L$ в области $D$, описываемой уравнениями:
\[
x = x(u_0, v), \quad y = y(u_0, v)
\]
Аналогично, линия $v = v_0$ соответствует кривой $l$ в области $D$:
\[
x = x(u, v_0), \quad y = y(u, v_0)
\]

Так как отображение взаимно однозначно, то через каждую точку $(x,y) \in D$ проходит единственная линия каждого семейства, соответствующая значениям $u = u_0$, $v = v_0$. Поэтому значения $(u,v)$ однозначно определяют точку $(x,y) \in D$. Придавая $u$ и $v$ допустимые значения, на плоскости $Oxy$ получим два семейства координатных линий. Значения $u$ и $v$ называются криволинейными координатами точки $(x,y) \in D$.

Частным случаем криволинейных координат является полярная система координат с началом в точке $O$. Переход от полярных координат к декартовым задаётся формулами:
\[
x = r\cos\varphi, \quad y = r\sin\varphi
\]
где $r \geq 0$ --- полярный радиус, $\varphi$ --- полярный угол.

\end{document}